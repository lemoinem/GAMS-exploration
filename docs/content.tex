\section*{Presentation}

GAMS provides an interface to many solvers and various facilities to manipulate
the results of these solver. However, incremental exploration of sets of indices
are not well supported. The goal of this script and library is to provide such
facilities.

This documention describes the use of the shell script
\texttt{GAMS-exploration.lisp}. The library documentation is provided directly
in the source code.

It is important to note that this script currently support only the SBCL
implementation of common-lisp, more about this restriction in the Restriction
Section.

\section*{Licence and Copyright}
The script, library and their documentation are licensed under the 3-clause BSD
License:\\
Copyright \textcopyright{} 2010, Mathieu Lemoine\\
All rights reserved.\\
% 
Redistribution and use in source and binary forms, with or without modification,
are permitted provided that the following conditions are met:
\begin{itemize}
\item Redistributions of source code must retain the above copyright notice,
  this list of conditions and the following disclaimer.
\item Redistributions in binary form must reproduce the above copyright notice,
  this list of conditions and the following disclaimer in the documentation
  and/or other materials provided with the distribution.
\item Neither the name of Mathieu Lemoine nor the names of contributors may be
  used to endorse or promote products derived from this software without
  specific prior written permission.
\end{itemize}
% 
This software is provided by Mathieu Lemoine ''as is'' and any express or
implied warranties, including, but not limited to, the implied warranties of
merchantability and fitness for a particular purpose are disclaimed. In no event
shall Mathieu Lemoine be liable for any direct, indirect, incidental, special,
exemplary, or consequential damages (including, but not limited to, procurement
of substitute goods or services; loss of use, data, or profits; or business
interruption) however caused and on any theory of liability, whether in
contract, strict liability, or tort (including negligence or otherwise) arising
in any way out of the use of this software, even if advised of the possibility
of such damage.

\section*{Syntax}

The syntax of the various files is designed to be as close as possible to the
syntax of ordinary GAMS model files.

\subsection*{Command-line arguments}

The script requires the following 9 mandatory arguments :
\begin{itemize}
\item \texttt{-{}-initial-points}: directory where initial points will be
  stored. It must end with a trailing slash and be an existing directory;
\item \texttt{-{}-sets}: list of sets of indices which are to be
  incrementally explored;
\item \texttt{-{}-stops}: list of stop criteria used to stop exploration of
  the sets;
\item \texttt{-{}-strategies}: list of strategies to be used to explore the dynamic
  sets;
\item \texttt{-{}-gms}: GAMS model;
\item \texttt{-{}-variables}: list of variables in the GAMS model;
\item \texttt{-{}-current-sets}: file included in the GAMS model to load the
  dynamic sets. This file does not need to exists since it will be generated by
  the script;
\item \texttt{-{}-current-point}: file included in the GAMS model to generate the
  start point. This file does not need to exists since it will be rewritten by
  the script;
\item \texttt{-{}-solvers}: list of solvers to be used.
\end{itemize}
The following optional arguments are also available:
\begin{itemize}
\item \texttt{-{}-lsts}: directory in which the lst generated by GAMS will be
  stored;
\item \texttt{-{}-min}: file in which will be stored the minimal solution found so
  far;
\item \texttt{-{}-max}: file in which will be stored the maximal solution found so
  far;
\item \texttt{-{}-errors}: file in which the list of GAMS errors will be stored.
\end{itemize}

This list is printed by the script itself if the argument \texttt{-{}-help} is
given. In this case, the script terminate after printing the list.

\subsection*{Dynamic sets file}

The dynamic sets file should contains a comma/new-line separated list of set descriptors.
A set descriptor is as follow:
\begin{center}
  \texttt{set-name[([start-size,[min-size,]]max-size)] [comment]}
\end{center}

\subsection*{Stop criteria file}

The stop criteria file is the only one not respecting a syntax close to GAMS's
files. This file usualy needs only to exist and be empty unless you need to
specify advanced stop criteria. The stop criteria must be written in Common-Lisp
and using the helper macros \texttt{sets-max-size}, \texttt{sets-min-size} and
\texttt{def-stop-criterion} in \texttt{backend/stop-criteria.lisp}. These macros
documentation is available directly in the source code.

\subsection*{Strategies file}

The strategies file contains a comma/new-line separated list of strategy
descriptors and strategies domain descriptors. Each strategy descriptor
represent a strategy that will be used to generate new initial points during the
dynamic sets exploration.  Each strategy domain descriptor represent a bound on
the values of set points for which the strategy will be used. A strategy
descriptor is as follow:
\begin{center}
  \texttt{strategy-file-name[(derivation [, set])] [comment]}
\end{center}
Since the value of \texttt{derivation} is a literal, it should be enclosed in
single quotes. A strategy domain descriptor is as follow:
\begin{center}
  \texttt{strategy-file-name.[lo|up](set) = bound;}
\end{center}

The strategy-file-name file must be written in GAMS, its content will be
included in the model to complete the initialization of the initial-point, as if
included using:\\
\texttt{\$batinclude strategy-file-name [new-set-element
  [old-set-element]]}. \texttt{new-set-element} will be present only if the
derivation is \texttt{derived} or \texttt{family} and will be the element that
has just been added in the currently explored dynamic set and
\texttt{old-set-element} will be present only if the derivation is
\texttt{family} and will be the element of the current set that is currently
used to generate a new point.\\
\texttt{derivation}, if present, must be one of:
\begin{itemize}
\item \texttt{i[ndependent]}, this is the default;
\item \texttt{d[erived]};
\item \texttt{f[amily]}.
\end{itemize}
\texttt{set}, if present, must be the name of a dynamic set. In a strategy
descriptor, \texttt{set} is the dynamic set that the strategy uses to derive new
points. To derive a point from previously generated results (for
\texttt{derived} and \texttt{family} derivations), a pool of previous result
points must be chosen. These points are the ones associated to an exploration
step using the same set sizes as the current step but for the size of the
\texttt{set} specified in strategy descriptor, which is reduced by 1.\\
\texttt{bound} must be a non-negative integer. It is a bound on the size of the
\texttt{set} of the strategy domain descriptor. A strategy is used for an
exploration point if all the set sizes satisfy the bounds required by the
strategy. That is if they are greater than or equal to the associated lower
bound and less than or equal to the associated upper bound. If the domain
defined for a strategy is empty, the strategy will simply never be used.

\subsection*{GAMS model}

The \texttt{GAMS-model.gms} is a regular GAMS model. No major restriction apply
to it. However, for the script to function, two files must be included within
the model:
\begin{itemize}
\item the \texttt{set-point.inc} file must be included before the use of any
  dynamic set in the model file;
\item the \texttt{initializer.inc} file must be included before the
  \texttt{solve} statement.
\end{itemize}
Moreover, the set element \texttt{null} is declared as \texttt{\$phantom} and
only one solve statement should be included in the model. Otherwise, only the
point generated by the last solve statement will be stored unless any solve
statement generates no data (in which case no result point will be stored).

\subsection*{Variables list file}

The \texttt{variables.inc} file is a regular GAMS file presenting the list of
variables to be stored when a result point is stored. The variables not present
in this file or present on lines not starting by a space are ignored.

\subsection*{Solvers file}

The \texttt{solvers} file should contain a list of solver options to use. Each
line may be either \texttt{default} or a set of options to change the default
GAMS sovlers for any kind of program.

\subsection*{Generated files}

The \texttt{set-point.inc} and \texttt{initializer.inc} files are generated by
the script. Their names should be specified on the command-line so the user is
free to name them as he want.

\section*{Behavior}

The script explores the dynamic sets space using a breadth-first
exploration. \texttt{derivation} determines how the new point is generated by
the strategy:
\begin{itemize}
\item \texttt{independent} means that the strategy does not required any a
  priori information to generate a new point;
\item \texttt{derived} means that the strategy generates a new initiail point by
  slightly modifying a previous exploration result point. The point will be
  loaded in GAMS before the strategy file is included using \texttt{\$batinclude
    strategy-file-name new-set-element};
\item \texttt{family} means that the strategy generates a new initial point by
  slightly modifying a previous exploration result point for each element in the
  current set during the previous exploration step. These strategies are used to
  generate a new point for each feasible point generated by and each element is
  the currently explored set during the previous exploration step. The point
  will be loaded in GAMS before the strategy file is included using
  \texttt{\$batinclude strategy-file-name new-set-element old-set-element}.
\end{itemize}

The initial exporation point is created by assigning to each set its start size
(which is 0 if unspecified). For each exploration point to explore, a set of
strategies are used to generate initial points (Unless a stop criterion is
reached). If a strategy is using a \texttt{derived} or \texttt{family}
derivation, a pool of previously generated result points to modify is
required. These points are the ones associated to the exploration point
identical to the one to-be-explored but for the coordinate of the \texttt{set}
specified in strategy descriptor which is reduced by 1.  After the set-point has
been explored, new set-points obtained by adding an element to each set is added
to the list of set-point to explore.

When no more set-point remains to be explored (e.g. because they all reached a
stop-criterion), the last set reaches stop criteria, the exploration is stoped
and the best and worst feasible points (actually, minimum and maximum) are
printed on the screen. A list of the errors generated by GAMS is also
printed. The initial points files are all in the initial-points
directory. Moreover, the lst file of each GAMS run is preserved if the
\texttt{-{}-lsts} arguments is specified.

\section*{Restrictions}

The script depends on the following libraries:
\begin{itemize}
\item \texttt{alexandria}: \url{http://common-lisp.net/project/alexandria/};
\item \texttt{cl-fad}: \url{http://weitz.de/cl-fad/};
\item \texttt{cl-ppcre}: \url{http://www.cliki.net/cl-ppcre};
\item \texttt{split-sequence}: \url{http://www.cliki.net/parse-number};
\item \texttt{script-utility}: No URL is currently available, please contact the
  script author.
\item \texttt{priority-fifo}: No URL is currently available, please contact the
  script author.
\end{itemize}

As mentionned in the presentation, this script currently supports only SBCL.
The details of this restriction are:
\begin{itemize}
\item the library \texttt{script-utility} currently supports only SBCL;
\item the script is run using the sbcl script interpreter;
\item the function \texttt{solve-GAMS-model} in \texttt{frontend/GAMS.lisp}
  currently supports only SBCL.
\end{itemize}
